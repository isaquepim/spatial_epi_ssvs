\begin{dedicatoria}
    \vspace*{\fill}
    %\noindent
    \hfill
    \begin{minipage}{.6\textwidth}
     Dedico essa dissertação aos meus pais: Marcelo e Munira. 
    \end{minipage}
\end{dedicatoria}
 
\begin{agradecimentos}

    Agradeço meus pais, Marcelo e Munira, por razões intermináveis e que não caberiam no escopo desse agradecimento, agradeço por todo amor que nunca me faltou. E meu irmão, Miguel, companheiro e amigo à longa distância. 
    
    Agradeço meus companheiros de turma por toda jornada até aqui. Em especial, meus companheiros do CDMC, que fizeram parte ativa de minha vida no Rio de Janeiro. E meus colegas de tempos passados, que ou ficaram no Espírito Santo, ou também tomaram outros rumos, mas ainda mantém contato
    
    Agradeço o Centro para o Desenvolvimento de Ciências e Matemática (CDMC) da FGV por acreditarem no meu potencial, e custearem todo processo de graduação.
    
    Agradeço meus professores da graduação em Matemática Aplicada da FGV por toda disposição e contribuição com meu crescimento ao longo do curso. 
    
    Destes, agradecimentos especiais ao meu orientador, Luiz Max Carvalho, por não ter faltado em momento algum no desenvolvimento deste trabalho, por todo direcionamento e gama de referências, que certamente me fizeram um acadêmico melhor. E claro, agradeço sua eficiência invejável na comunicação por e-mail.
    
    
    
    
    

\end{agradecimentos}

\begin{epigrafe}
\vspace*{\fill}

\begin{flushright}
    \hspace{7.5cm}
    \textit{
        ``Eu vi uma flor\\
pendida de um galho,\\
banhada de orvalho\\
das noites de abril.\\[5pt]
Se eu fosse um pintor,\\
artista de fato,\\
faria um retrato\\
de flor tão gentil.\\[5pt]
Estava no meio,\\
por entre a folhagem \\
da verde ramagem \\
um tanto escondida.\\[5pt]
Talvez com receio\\
que mãos criminosas,\\
almas venenosas,\\
tirassem- lhe a vida.''} \\
        \textbf{Albércio Vieira Machado}
\end{flushright}
\end{epigrafe}