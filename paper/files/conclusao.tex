\chapter{Conclusão}

Modelos hierárquicos Bayesianos são um forte aliado de qualquer pesquisador da área de epidemiologia. Uma de suas grandes vantagens é ser altamente interpretável, e poder acomodar todo tipo de hipótese dentro de sua formulação. Em especial, para epidemiologia espacial conseguimos acoplar a necessidade por dependência espacial, que está presente em todas as coisas, como diz a Primeira Lei da Geografia de \cite{Tobler1970}. Conseguimos acoplar também uma análise de regressão, uma poderosa ferramente estatística, mas que deve ser tratada com cautela. Neste trabalho tivemos complicações com o modelo do Ebola, não obtendo a desejada convergência da cadeia. Apesar do modelo ter tido bons resultados em termos de RMSE, resultados estatísticos não podem ser traçados. 

Possíveis correções e sugestões para futuros trabalhos são: analisar diferentes amostradores e sua eficiência comparada aos amostrados base do NIMBLE. O próprio NIMBLE já possui em versão beta amostradores por HMC (Hamiltonian Monte Carlo). Outra opção é o INLA (INtegrated Nested Laplace Approximation ). Mesmo que tivessemos conseguido um bom amostrador, temos o problema da alta correlação entre as covariáveis do ebola, que pode distribuir a importância dessas variáveis e levar à interpretação errada de parâmetros. Algumas sugestões são adicionar ao modelo através das prioris, estruturas que permitam a agregação de variáveis correlacionadas, como em \cite{IsingDP}. Uma terceira indagação para o futuro é sobre quais seriam os melhores diagnósticos de cadeia para este cenário. Veja que no conjunto de dados do Ebola, apesar de não termos atingido convergência da cadeia, obtivemos bons resultados de RMSE, o que poderia levar a conclusões erradas sobre a eficácia do modelo. 